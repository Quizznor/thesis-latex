%! TEX root = ../main.tex

\titleformat{\chapter}[display]{\centering\normalfont\huge}{}{0pt}{#1}[\hrule\vspace{-0.5cm}]
\printbibheading

The Pierre Auger observatory hosts an internal database of papers. These 
typically short reports serve to accelerate the exchange of knowledge within the
collaboration, and are called \GAP notes. Since they contain information that is
not freely accessible outside the Pierre Auger collaboration, they are listed in
a special category with the prefix \textbf{\textit{A}} alongside other 
information, that is accessible only to members of the Auger collaboration. 
Similarly, non-public sources from personal correspondence are grouped with the 
prefix \textbf{\textit{C}}. All public (non-physics related) information has the 
label \textbf{\textit{O}}. Finally, physical references, also containing 
publications by the Pierre Auger collaboration can be found with label 
\textbf{\textit{P}}.

% Auger Internal References
\newrefcontext[labelprefix=A]
\printbibliography[keyword={auger},title={Auger internal},heading=subbibliography]

% Personal Correspondence
\newrefcontext[labelprefix=C]
\printbibliography[keyword={private},title={Personal Correspondence},heading=subbibliography]

% Other References
\newrefcontext[labelprefix=O]
\printbibliography[notkeyword={private},notkeyword={phys},notkeyword={auger},title={Other References},heading=subbibliography]

% General Physical References
\newrefcontext[labelprefix=P]
\printbibliography[keyword={phys},notkeyword={auger},title={Physics References},heading=subbibliography]

\cleardoublepage

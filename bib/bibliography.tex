%! TEX root = ../main.tex

\titleformat{\chapter}[display]{\centering\normalfont\huge}{}{0pt}{#1}[\hrule\vspace{-0.5cm}]
\titleformat{\section}[display]{\bfseries\Large}{}{10pt}{\sc{#1}}[]
\printbibheading

\defbibnote{auger}{%
The Pierre Auger observatory hosts an internal database of papers. These 
typically short reports serve to accelerate the exchange of knowledge within the
collaboration, and are called \GAP notes. Since they are not freely accessible 
outside the Pierre Auger collaboration, they are listed in a special category 
with the prefix \sc{A}\normalfont, alongside other information that is 
available only to Auger members.}

\defbibnote{private}{%
Non-public sources from personal correspondence are grouped under the prefix 
\sc{C}.}

\defbibnote{other}{%
All public (non-physics related) reference are marked with the label \sc{O}.}

\defbibnote{physics}{%
Physical references, including publications from the \PAO, can be found under 
the label \sc{P}.}

% Auger Internal References
\newrefcontext[labelprefix=A]
\printbibliography[keyword={auger},title={Auger Internal Documents},heading=subbibliography, prenote={auger}]
\vspace{0.5cm}\hrule\vspace{0.2cm}

% Personal Correspondence
\newrefcontext[labelprefix=C]
\printbibliography[keyword={private},title={Private Communication},heading=subbibliography, prenote=private]
\vspace{0.5cm}\hrule\vspace{0.2cm}

% Other References
\newrefcontext[labelprefix=O]
\printbibliography[notkeyword={private},notkeyword={phys},notkeyword={auger},title={Other References},heading=subbibliography, prenote=other]
\vspace{0.5cm}\hrule\vspace{0.2cm}

% General Physical References
\newrefcontext[labelprefix=P]
\printbibliography[keyword={phys},notkeyword={auger},title={Physics References},heading=subbibliography, prenote=physics]
\vspace{0.5cm}\hrule\vspace{0.2cm}

\cleardoublepage

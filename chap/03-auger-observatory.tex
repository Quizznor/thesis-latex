%! TEX root = ../main.tex 

\chapter{The Pierre Auger Observatory}
\label{chap:pierre-auger-observatory}

The \PAO is the largest scientific experiment in the world, spanning an area of
roughly \SI{3000}{\kilo\meter\squared}. It consists of an array of 1660 \WCDs, 
which form the \SD, and 27 fluorescence telescopes, that comprise the \FD. The
simultaneous detection of \CR-artifacts in the air and on ground via this hybrid
approach offers a unique possibility to observe \UHECRs at the tail-end of the 
\CR energy spectrum.

%We begin this chapter in \cref{sec:science-case} by formulating open questions 
%that the \PAO aims to answer. Design details for the \FD and for the \SD are 
%given in \cref{sec:fd} and \cref{sec:sd} respectively. After a discussion of the
%local \DAQ processes and the centralized event detection in \cref{sec:cdas}, we 
%finish by detailing the procedure of the event reconstruction and higher level 
%analysis in \cref{sec:rec}.

\section{Project History and Future}
\label{sec:project-history-and-future}

\subsection{Science Goals and Achievements}
\label{ssec:science-goals-and-achievements}

The project that is today known as the \acl{PAO} looks back on a history which 
spans more than three decades in total. First plans for a \CR detector of
immense exposure were devised by Nobel laureate James Cronin 
\cite{nobelprizeoutreach2025NobelPrizePhysics} and Faraday medalist Alan Watson 
\cite{FellowWinsIoP2011} at the 22nd \ICRC \cite{watsonDevelopmentPierreAuger}.

They realized that the flux of \UHECRs is very low 
($\approx\SI{1}{\per\km\squared\per\year}$ for energies above \SI{1e19}{\eV} 
\cite{fenuCosmicRayEnergy2023}), and that one needs to observe a large area over
a long timespan to collect enough data for a statistically relevant analysis of 
their dynamics. Cronin especially was interested in  verifying the possible 
detection of an astrophysical source of $\gamma$-rays 
\cite{samorskiDetection2101983}. Moreover, the mere existence of some \UHECRs 
was challenging astrophysical models \cite{birdDetectionCosmicRay1995, 
naganoAstrophysicalAspectsMost1991, hayashidaObservationVeryEnergetic1994,
lawrenceCosmicRayEnergy1991}. Naturally, probing the presence and possible 
sources of such \CRs of extreme energy became a priority for the envisioned 
experiment.

Cronin and Watson gathered support for their ideas and, in 1995, founded a 
collaboration of 140 likeminded scientists in 17 participating countries. A 
white paper was published that outlined the construction, capabilities, and 
cost of the \PAO \cite{theaugercollaborationPierreAugerObservatory}. The 
detector configuration presented in the document varies from the design in use 
today. Most importantly, what was originally envisioned to be a detector with an
active area of \SI{6000}{\km\squared} spread over two locations (Auger North and
Auger South) had to be amended and only one site with and area of 
\SI{3000}{\km\squared}, located in the Argentinian Pampa Amarilla in the Mendoza
province, could be realized. There, construction of an Engineering Array - a 
full scale prototype of the first \SD and \FD detectors (see \cref{sec:sd} and 
\cref{sec:fd} respectively) was completed in 2001. More hardware was added after
first successful measurements, making the \PAO the largest observatory for \CRs 
by the end of 2003.

The detector has been operating almost continuously ever since, resulting in a 
total exposure well surpassing \SI{100000}{\km\squared\sr\year} 
\cite{aabPierreAugerObservatory2020}. Thanks to this, the Pierre Auger 
Collaboration has delivered key insights that advanced our understanding of 
astroparticle and high-energy physics. Most notably, the \PAO confirmed a 
strong suppression of the \CR-flux for particles above 
$E\approx\SI{4e19}{\eV}$ \cite{yamamotoUHECRSpectrumMeasured2007} (c.f. 
\todo{ref physics chapter}). Anisotropies in the arrival directions of \UHECRs 
hint to their distant origins
\cite{thepierreaugercollaborationObservationLargescaleAnisotropy2017} (c.f. 
\todo{ref physics chapter}). Upper limits have been set on the \UHE photon and
neutrino flux that force the reevaluation of certain astrophysical models 
\cite{collaborationPierreAugerObservatory2011, abreuSearchUltrahighEnergy2011}.
Last but not least, regular releases of datasets to the public support the work 
of independently working astrophysicists around the globe 
\cite{pierreaugercollaborationPierreAugerObservatory2025}.

\subsection{AugerPrime and Outlook}
\label{ssec:augerprime-and-outlook}

Many questions have been answered thanks to the data collected by the \PAO. It
is no surprise however that new insights sprout new questions. The existence of 
\UHECRs beyond the GZK-cutoff has been established unambigously. Now, their
nature must be studied more deeply. Of particular interest is the reconstruction
of the primary particle mass from \EAS-data, as this influences the \CRs
(reconstructed \cite{yushkovMassCompositionCosmic2021}) energy, and evolution 
from the source to the earth \cite{strongCosmicRayPropagationInteractions2007, 
flaggsStudyingMassSensitivity2024}.

The detectors have undergone a major upgrade in the past decade in order to 
increase the observatories sensitivity to the primary particle mass. This 
overhaul, called AugerPrime, equips the \SD stations with additional detector 
channels and improved readout electronics. The \FD telescopes duty cycle was 
improved, and new methods for data analysis are being tested.

As of the completion of this thesis, the AugerPrime hardware and software 
upgrade is completed and described in \todo{augerprime ref here} and 
\cite{collaborationPierreAugerObservatory2011}. The observatory delivers data 
with lower systematics and additional information from the new detectors. The 
prospects of this detector upgrade have astroparticle physicists excited, and 
resulted in the strong support to extend the observatories funding until beyond
the year 2030 \cite{castellinaOutcomeFinanceBoard2023}. The phase II of \DAQ (as
opposed to Phase I pre- and during AugerPrime) will be the basis for many 
contributions that attempt to shed light into the physics of \UHECRs, with this 
work being just one example.

\section{The Fluorescence Detector}
\label{sec:fd}

The Fluorescence Detector of the \PAO is a set of 27 reflector telescopes tuned
to detect faint sources of \UV light. More specifically, the aim of the \FD is
to observe the \UV-emission of \EAS (c.f. \todo{physics ref here}). However, 
since the solar irradiance ($\approx\SI{120}{\watt\per\meter\squared}$ 
\cite{leanContributionUltravioletIrradiance1989}) and even the lunar irradiance 
($\approx\SI{16}{\nano\watt\per\meter\squared}$
\cite{snowAbsoluteUltravioletIrradiance2013}) in the \UV-band far outshine the 
emission of \UV-light by cosmic rays ($<\SI{1}{\nano\watt\per\meter\squared}$, 
c.f. \cref{app:cr-uv-irradiance}), the \FD can only operate after astronomical 
twiglight, and when the moon is not directly in the telescope \FOV, and not
more 70\% illuminated \cite{mathesCriteriaFDShift}. This drops the duty cycle 
to approximately 13\% \cite{abrahamFluorescenceDetectorPierre2010}.

hallo


Ignoring three exceptions (see \autoref{ssec:heat}), all telescopes are grouped
at four \FD sites, where a collection of six identical setups offer a $180^\circ
\times30^\circ$ view (Azimuth $\times$ Elevation) over the \SD array. 
\autoref{fig:auger-map} shows the location of these sites relative to the \SD.
Commonly, these sites are referred to via their names \CO, \LA, \LM, and \LL.

\begin{figure}[t]
  \centering
  \subfloat[]{\includegraphics[width=0.48\textwidth]{auger-observatory/auger_array-small.png}
  \label{fig:auger-map}
  }\hspace{0.2cm}
  \subfloat[]{\includegraphics[width=0.48\textwidth]{auger-observatory/fd_schematic.png}
  \label{fig:fd-schematic}
  }
  \caption[]{\subref{fig:auger-map} Overview of the \PAO. The black dots give the 
  location of \WCDs. The blue lines indicate the location and FOV of the \FD. 
  HEAT (red) overlooks the Infill area of the \SD-array. From 
  \cite{veberic_index_nodate}. \subref{fig:fd-schematic} A schematic of a single
  telescope, or bay, of an \FD building, modified from 
\cite{abraham_fluorescence_2010}.}
  \label{fig:pao-images}
\end{figure}


\subsection{Telescope design}
\label{ssec:fd-design}

The \FD telescopes are designed following the schematic of a Schmidt camera with 
corrector plates. The individual components located on the optical axis are, in 
the order that light traverses them:

\begin{itemize}
  \item \textbf{Shutter:} \\
  The shutter consists of two massive metal doors that block the light 
  propagation. Several mechanisms automatically close the shutters to shield the 
  telescope from e.g. atmospheric influences like wind and rain, or an 
  unacceptably high light flux.
  \item \textbf{UV Filter:} \\
  A MUG6 UV-filter is installed in the light path in order to block visible 
  light from entering the telescope, and potentially damaging the sensitive 
  electronics. It separates the climate controlled inside of the \FD buildings 
  from the outside.
  \item \textbf{Corrector:} \\
  A ring of aspheric corrector lenses is located at the edge of the aperture,
  the ring has an inside diameter of \SI{1.7}{\meter} and serves to increase 
  the light collection area, while minimizing spherical aberration in the 
  camera.
  \item \textbf{Curtain:} \\
  The curtain provides a secondary failsafe, that prevents high-intensity light 
  from reaching the electronics. During nominal operation the curtain sits outside 
  of the light path. It can be deployed in emergency situations, or during daytime
  work and maintenance.
  \item \textbf{Mirror:} \\
  The mirrors are a set of hexagonally (square, in the case of \LL) segmented 
  mirrors aligned in a concave $\SI{3.6}{\meter}\times\SI{3.6}{\meter}$ shape, 
  which reflects and concentrates incoming light towards the camera.
  \item \textbf{Camera:} see next subsection.
\end{itemize}

A schematic overview of the setup in each telescope bay is given in
\cref{fig:fd-schematic}.

\subsection{Camera and Readout Electronics}

The camera is made up of 440 highly sensitive \PMTs that make up the individual
pixels. They are arranged in 22 rows and 20 columns. Each pixel has a solid angle 
acceptance of $\Omega = \SI{5.94e-4}{\sr}$. Reflective star shapes with threefold 
symmetry are mounted between pixels. They are commonly referred to as Mercedes 
stars, due to their characteristic shape, and increase both light collection as 
well as ensure a good point resolution between adjacent pixels.

The pixel data is read out at a sampling frequency of \SI{10}{\mega\hertz} 
(\SI{20}{\mega\hertz} in the case of \HE). \todo{continue}


\subsection{High Elevation Auger Telescope (HEAT)}
\label{ssec:heat}

Three additional telescopes, called \HEs are located close to the Coihueco site.
The construction of these telescopes is identical to the telescopes discussed
above. However, in the case of \HE, the entire telescope frame can be rotated,
which allows the observation of the upper atmosphere, and effectively extends the 
energy scale seen by the \FD to lower energies.


\subsection{Telescope Calibration}
\label{ssec:fd-calibration}

The UV-telescopes observe the longitudinal shower profile as an EAS develops. As 
shown in \autoref{chap:cosmic-rays}, this enables the \FD to estimate the total 
energy of the primary particle, by relating it to the count of photons observed 
in the telescope.

The individual pixel PMTs show a drop \todo{is this correct?} in voltage when exposed to UV light. This
voltage gets converted to \ADC via \todo{what?}. The exact \ADC value that a 
given voltage corresponds to is completely determined by the PMT gain, and
fluctuates over the course of a measurement night. Thus in order to obtain a 
physically meaningful observable the conversion factor of \ADC $\rightarrow$ 
photon number needs to be known. This is achieved by illuminating the \FD camera 
with a light source of known intensity, and measuring the camera response. 

\subsubsection{Drum calibration}

The first absolute calibration of the \FD telescopes is the Drum calibration. 


\subsubsection{XY-scanner}



\subsubsection{Calibration A}

In order to track the relative drift between absolute calibration campaigns, an 
LED located in the center of the mirror shoots a known amount of light directly 
at the \FD camera. It is performed twice each night, once before \DAQ, and once 
more after \DAQ.

The importance of this so-called Calibration A becomes apparent when examining the
response of the \FD cameras, as e.g. shown in \autoref{fig:calibration-a-signal}.
Comparing measurements across a larger timespan, it can be seen that the Cal A
before \DAQ systematically has less recovered signal, i.e. measures fewer \ADC
than the post-\DAQ run, while in both runs the cameras are exposed to the same 
amount of light. Moreover, the recovered signal after extended downtimes is lower
than at the end of a measurement period.

This is a consequence of the \PMT electronics being turned off between measurment 
cycles. Each time this occurs, \PMTs first have to warm up again, until full
collection efficiency for incoming photons is reached.

\todo{cal a plot}


\section{The Surface Detector}
\label{sec:sd}

\todo[inline]{roughly mention design, duty cycle}

\section{Central Data Acquisition System (CDAS)}
\label{sec:cdas}



\section{\Offline and Event Reconstruction}
\label{sec:rec}

\begin{figure}[t]
  \centering
  \subfloat[]{\includegraphics[width=0.47\textwidth]{cosmic-rays/spectrum-double1.png}
  \label{fig:LABEL}
  }\hspace{0.2cm}
  \subfloat[]{\includegraphics[width=0.47\textwidth]{cosmic-rays/spectrum-double2.png}
  \label{fig:LABEL}
  }
  \caption[]{\subref{fig:LABEL} asdasd \subref{fig:LABEL} asdasd}
  \label{fig:}
\end{figure}

\begin{figure}[t]
  \centering
  \includegraphics[width=0.9\textwidth]{cosmic-rays/spectrum-single.png}
  \caption{asdasdasdasd}
  \label{fig:asdasd}
\end{figure}

\cref{chap:pierre-auger-observatory} 

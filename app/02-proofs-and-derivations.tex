%! TEX root = ../main.tex

\chapter{Derivations and Calculations}
\label{app:derivation}


\section{Approximate max. UV Irradiance of an EAS}
\label{app:cr-uv-irradiance}

We estimate the UV irradiance, $I_\text{UV}$, from fluorescence light stemming 
from an extensive air shower. To illustrate that telescopes need to be 
incredibly sensitive to observe this phenomenon, we assume fantastic to 
unrealistically good conditions for the \UV light yield and related parameters. 
This results in an optimistic estimate of $I_\text{UV}$.

We consider an \EAS of energy $E_0 = \SI{e18}{\eV}$ and inclination 
$\uptheta=\SI{15}{\degree}$. We assume the behaviour of the particle cascade 
resulting from the deeply inelastic is determined by the Heitler-Matthews model 
(see \todo{write this} and \cite{matthewsHeitlerModelExtensive2005, 
rissePhotonAirShowers2006}). This implies that hte atmospheric depth (height), 
at which the shower reaches the maximum multiplicity is

\begin{equation}
\label{eq:irradiance-multiplicity}
\Xmax \approx \SI{700}{\gram\per\centi\meter\squared}\qquad\left(\;\widehat{\approx}\;\SI{18.5}{\kilo\meter}\;\text{above earth surface}\footnotemark\right).
\end{equation}
\footnotetext{To calculate the height above surface we assume a purely 
isothermic atmosphere with $T=\SI{278}{\kelvin}$, and a density of 
$\rho=\SI{0.86}{\kilogram\per\meter\cubed}$ at an altitude of \SI{1400}{\meter} 
above sea level.}

At $\Xmax$, the EM component of the shower contains roughly
$1-\left(E_0\,/\,\xi^\pi_C\right)^{\beta-1}\approx92\%$ of the primary particle 
energy. Photons and electrons share the energy to roughly equal parts. It
follows that $E_\text{UV}=0.92\cdot0.5\cdot\SI{e18}{\electronvolt}=
\SI{4.6e17}{\electronvolt}$ are available a priori to create fluorescence light.
\cite{keilhauerNitrogenFluorescenceAir2013} gives an optimistic fluorescence 
light yield $\text{FY}\approx8\gamma\,/\,\SI{}{\mega\electronvolt}$. The 
majority of the produced light stem from the 2P(0,0) transition of N$_2$ 
\cite{aveSpectrallyResolvedPressure2008}, which has a characteristic wavelength 
of \SI{337.1}{\nano\meter}, and an average radiation time of 
\SI{42}{\nano\second} \cite{leanContributionUltravioletIrradiance1989}. We 
assume that all available energy $E_\text{UV}$ is immediately converted to 
molecular excitations of N$_2$ at $\Xmax$, and then gradually released according
to an exponential decay and recover an expression for the (UV) power output of 
the air shower.

\begin{equation}
\label{eq:irradiance-power}
P(t)= \text{FY}\,E_\text{UV}\,\frac{hc}{\SI{337.1}{\nano\meter}} \cdot\frac{e^{-\SI{42}{\nano\second}/t}}{t}   
\end{equation}

The maximum power output is thus $P_\text{max}\approx\SI{19.4}{\watt}$. We 
convert the maximum power output to an irradiance measured on ground and account
for the attenuation of \UB-light in the atmosphere. Specifically, light of 
wavelength \SI{337}{\nano\meter} drops in intensity by about 
\SI{13}{\percent\per\kilo\meter}  in clear conditions 
\cite[see Fig. 83 on page 103 of][]{baumAttenuationUltravioletLight2008}. 
Assuming that an observer on ground must be at least $d=\SI{18.5}{\kilo\meter}$ 
away from the $\Xmax$ (compare \cref{eq:irradiance-multiplicity}), the total 
irradiance that is measured reads

\begin{equation}
\label{eq:irradiance}
I_\text{UV} = P_\text{max}\,\frac{e^{-0.14\,\frac{d}{\SI{}{\kilo\meter}}}}{4\pi\,d^2}\approx\SI{0.32}{\nano\watt\per\square\meter}.
\end{equation}

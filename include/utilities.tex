\DeclareSIUnit\century{century}
\DeclareSIUnit\year{yr}
\DeclareSIUnit\sr{sr}
\DeclareSIUnit\km{\kilo\meter}
\DeclareSIUnit\Wp{Wp}
\DeclareSIUnit\dBi{dBi}
\DeclareSIUnit[per-mode=reciprocal]\grammage{\gram\per\cm\squared}

\newcolumntype{P}[1]{>{\centering\arraybackslash}p{#1}}
\newcolumntype{M}[1]{>{\centering\arraybackslash}m{#1}}

\newcommand{\TTWO}{\ac{T2}\xspace}
\newcommand{\TTWOs}{\acp{T2}\xspace}
\newcommand{\TTHREE}{\ac{T3}\xspace}
\newcommand{\TTHREEs}{\acp{T3}\xspace}

\newcommand{\DX}{\Delta X\xspace}
\newcommand{\exposuresym}{\mathcal{E}}
\newcommand*{\Imip}{\ensuremath{I_\text{MIP}}\xspace}
\newcommand*{\Qmip}{\ensuremath{Q_\text{MIP}}\xspace}
\newcommand*{\Ivem}{\ensuremath{I_\text{VEM}}\xspace}
\newcommand*{\Qvem}{\ensuremath{Q_\text{VEM}}\xspace}
\newcommand*{\VemOffline}{\ensuremath{I^\text{WCD}_\text{histo}}\xspace}
\newcommand*{\VemOnline}{\ensuremath{I^\text{WCD}_\text{rate}}\xspace}
\newcommand*{\MipOffline}{\ensuremath{I^\text{SSD}_\text{histo}}\xspace}
\newcommand*{\MipOnline}{\ensuremath{I^\text{SSD}_\text{rate}}\xspace}

% particle component abbreviations
\newcommand{\cmu}{\ensuremath{\mu}\xspace}
\newcommand{\ceg}{\ensuremath{e\gamma}\xspace}
\newcommand{\cegmu}{\ensuremath{e\gamma(\mu)}\xspace}
\newcommand{\ceghad}{\ensuremath{e\gamma(\mathrm{had})}\xspace}

\newcommand{\Xmax}{X_\mathrm{max}\xspace}
\newcommand{\Xmaxmu}{X_\mathrm{max}^\mu\xspace}
\newcommand{\Xmaxp}{X_\mathrm{max}^\mathrm{p}\xspace}
\newcommand{\XmaxFe}{X_\mathrm{max}^\mathrm{Fe}\xspace}
\newcommand{\XmaxSd}{X_\mathrm{max}^\mathrm{SD}\xspace}
\newcommand{\Xmaxuniv}{X_\mathrm{max}^\mathrm{Univ}\xspace}
\newcommand{\XmaxFd}{X_\mathrm{max}^\mathrm{FD}\xspace}
\newcommand{\XmaxMC}{X_\mathrm{max}^\mathrm{MC}\xspace}

\newcommand{\Esd}{E_\mathrm{SD}\xspace}
\newcommand{\Erec}{E_\mathrm{rec}\xspace}
\newcommand{\Etrue}{E_\mathrm{true}\xspace}
\newcommand{\Emc}{E_\mathrm{MC}\xspace}
\newcommand{\Efd}{E_\mathrm{FD}\xspace}
\newcommand{\Euniv}{E_\mathrm{Univ}\xspace}

\renewcommand{\arraystretch}{1.1}  % größerer Zeilenabstand in tabellen


\def\Offline{\mbox{$\overline{\textrm%
{Off}}$\hspace{.05em}\protect\raisebox{.4ex}%
{$\protect\underline{\textrm{line}}$}}\xspace}

% --- Macro \xvec
\makeatletter
\newlength\xvec@height%
\newlength\xvec@depth%
\newlength\xvec@width%
\newcommand{\xvec}[2][]{%
  \ifmmode%
    \settoheight{\xvec@height}{$#2$}%
    \settodepth{\xvec@depth}{$#2$}%
    \settowidth{\xvec@width}{$#2$}%
  \else%
    \settoheight{\xvec@height}{#2}%
    \settodepth{\xvec@depth}{#2}%
    \settowidth{\xvec@width}{#2}%
  \fi%
  \def\xvec@arg{#1}%
  \def\xvec@dd{:}%
  \def\xvec@d{.}%
  \raisebox{.2ex}{\raisebox{\xvec@height}{\rlap{%
    \kern.05em%  (Because left edge of drawing is at .05em)
    \begin{tikzpicture}[scale=1]
    \pgfsetroundcap
    \draw (.05em,0)--(\xvec@width-.05em,0);
    \draw (\xvec@width-.05em,0)--(\xvec@width-.15em, .075em);
    % \draw (\xvec@width-.05em,0)--(\xvec@width-.2em,-.05em);
    \ifx\xvec@arg\xvec@d%
      \fill(\xvec@width*.45,.5ex) circle (.5pt);%
    \else\ifx\xvec@arg\xvec@dd%
      \fill(\xvec@width*.30,.5ex) circle (.5pt);%
      \fill(\xvec@width*.65,.5ex) circle (.5pt);%
    \fi\fi%
    \end{tikzpicture}%
  }}}%
  #2%
}
\makeatother

\makeatletter
\newlength\pt@height%
\newlength\pt@depth%
\newlength\pt@width%
\newcommand{\pt}[1]{%

\settoheight{\pt@height}{$#1$}\settodepth{\pt@depth}{$#1$}\settowidth{\pt@width}{$#1$}%
   \raisebox{.2ex}{\raisebox{\pt@height}{\rlap{\makebox[\pt@width][c]{%
     \kern.1em%
     \begin{tikzpicture}[scale=1]
       \draw(0,0) circle[radius=0.08em];
     \end{tikzpicture}%
   }}}}%
   #1
}
\makeatother

% --- Override \vec with an invocation of \xvec.
\let\stdvec\vec
\renewcommand{\vec}[1]{\xvec[]{#1}}
% --- Define \dvec and \ddvec for dotted and double-dotted vectors.
\newcommand{\dvec}[1]{\xvec[.]{#1}}
\newcommand{\ddvec}[1]{\xvec[:]{#1}}
\newcommand{\psl}[1]{#1\sp{\prime}}
\newcommand{\fold}[1]{#1\sp{\prime}}
% \renewcommand{\vec}[1]{\mathbf{#1}}
\newcommand{\uvec}[1]{\hat{#1}}

\renewcommand{\d}[1]{\ensuremath{\operatorname{d}\!{#1}}}
\newcommand{\dn}[2]{\ensuremath{\operatorname{d}^{#1}\!{#2}}}
\newcommand{\sig}[1]{\ensuremath{#1\,\sigma}}

\newcommand{\parnum}[1]{\num[round-mode=figures,round-precision=3,retain-explicit-plus,scientific-notation=fixed,fixed-exponent=0]{#1}}

\DeclareDocumentCommand{\rpol}{O{}mooooooo}{%
    \ensuremath{\parnum{#1} \parnum{#2} \, \hat{r} \IfNoValueTF{#3}{}{\parnum{#3} \, \hat{r}^2} \IfNoValueTF{#4}{}{\parnum{#4} \, \hat{r}^3} \IfNoValueTF{#5}{}{\parnum{#5} \, \hat{r}^4} \IfNoValueTF{#6}{}{\parnum{#6} \, \hat{r}^5} \IfNoValueTF{#7}{}{\parnum{#7} \, \hat{r}^6} \IfNoValueTF{#8}{}{\parnum{#8} \, \hat{r}^7} \IfNoValueTF{#9}{}{\parnum{#9} \, \hat{r}^8} }%
}

\def\avg#1{{\left\langle{#1}\right\rangle}}
\def\avgx#1{{\langle{#1}\rangle}}
\def\expect#1{\operatorname{E}[#1]\xspace}
\def\var#1{\operatorname{Var}[#1]\xspace}
\def\erf#1{\operatorname{erf} #1 \xspace}
\def\Pois{\operatorname{Pois}}

